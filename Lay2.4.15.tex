\documentclass{article}
\usepackage{amssymb,amsmath,chngcntr,graphicx,float,biblatex}

\begin{document}
\author{Timothy Smith}
\title{Lay Problem 1.9.4}
\maketitle

\counterwithin*{equation}{section}
\section*{Problem 2.4.15}

Suppose $A_{11} $ is invertible. Find X and Y such that

\[
\begin{pmatrix}
A_{11} & A_{12} \\
A_{21} & A_{22} 
\end{pmatrix}
=
\begin{pmatrix}
I & 0 \\
X & I 
\end{pmatrix}
\begin{pmatrix}
A_{11} & 0 \\
0 & S
\end{pmatrix}
\begin{pmatrix}
I & Y \\
0 & I
\end{pmatrix}
\]
where $ S= A_{22}-A_{21}A^{-1}_{11}A_{12}. $. The matrix  S is called the Schur complement of $A_{11}$. Likewise, if $A_{22}$ is invertible, the matrix $ A_{11}-A_{12}A^{-1}_{22}A_{21} $ is called the complement of $ A_{22}$. Such expressions occur frequently in the theory of systems engineering, and elsewhere.
\\
\\  
\begin{equation}
\begin{pmatrix}
A_{11} & A_{12} \\
A_{21} & A_{22} 
\end{pmatrix}
=
\begin{pmatrix}
I & 0 \\
X & I 
\end{pmatrix}
\begin{pmatrix}
A_{11} & 0 \\
0 & S
\end{pmatrix}
\begin{pmatrix}
I & Y \\
0 & I
\end{pmatrix}
\end{equation}

First we multiply the matrices 
\begin{equation}
\begin{pmatrix}
A_{11} & 0 \\
0 & S 
\end{pmatrix}
\begin{pmatrix}
I \\
0
\end{pmatrix}
=
\begin{pmatrix}
A_{11}I\\
0
\end{pmatrix}
\end{equation}
\begin{equation}
\begin{pmatrix}
A_{11} & 0 \\
0 & S 
\end{pmatrix}
\begin{pmatrix}
Y \\
I
\end{pmatrix}
=
\begin{pmatrix}
A_{11}Y\\
SI
\end{pmatrix}
\end{equation}
After multiplying the matrices our equation resembles
\begin{equation}
\begin{pmatrix}
A_{11} & A_{12} \\
A_{21} & A_{22} 
\end{pmatrix}
=
\begin{pmatrix}
I & 0 \\
X & I 
\end{pmatrix}
\begin{pmatrix}
A_{11} & A_{11}Y \\
0 & S
\end{pmatrix}
\end{equation}
We now multiply the two matrices on the right side of the equation to arrive at a singular matrix. 
\begin{equation}
\begin{pmatrix}
I & 0 \\
X & I 
\end{pmatrix}
\begin{pmatrix}
A_{11}\\
0
\end{pmatrix}
=
\begin{pmatrix}
A_{11} \\
XA_{11}
\end{pmatrix}
\end{equation}

\begin{equation}
\begin{pmatrix}
I & 0 \\
X & I 
\end{pmatrix}
\begin{pmatrix}
A_{11}Y\\
S
\end{pmatrix}
=
\begin{pmatrix}
A_{11}Y \\
XA_{11}Y+S
\end{pmatrix}
\end{equation}
\\
\\
We now arrive at the following equation

\begin{equation}
\begin{pmatrix}
A_{11} & A_{12} \\
A_{21} & A_{22} 
\end{pmatrix}
=
\begin{pmatrix}
A_{11}& A_{11}Y \\
XA_{11} & XA_{11}Y +S
\end{pmatrix}
\end{equation}
We now have the following resulatant
\begin{align*}
 A_{11}&=A_{11}  \\
A_{12}&=A_{11}Y \\
A_{21}&=XA_{11}\\
A_{22}&=XA_{11}Y+S
\end{align*}
Since $ A_{11} $ is invertible we can solve for both X and Y by multiplying by the inverse.
where 
$$ Y=A^{-1}_{11}A_{12}$$
and
$$ X=A_{21}A^{-1}_{11}$$
Solving for S and substituting in for X and Y we arrive at the Schur complement of $A_{11}$
 $$ S= A_{22}-A_{21}A^{-1}_{11}A_{12} $$














\end{document}