\documentclass{article}
\usepackage{amssymb,amsmath,chngcntr,graphicx,float,biblatex,tikz,physics}

\begin{document}
\author{Timothy Smith}
\title{Lay}
\maketitle

\counterwithin*{equation}{section}
\section*{Problem 5.3.28}
Show that if A  has n linearly independent eigenvectors, then so does $A^T$ [Hint: use the Diagonalization Theorem]
\\
\\
If A has n linearly independent eigenvectors thus A is diagonalizable and the columns of P are n linearly independent vectors of A. 
\begin{equation}
A=PDP^{-1}
\end{equation}
The transpose of this is 
\begin{equation}
A^T=(PDP^{-1})^T
\end{equation}
Now we can use the Theorem 3 d.) on page 101 of Lay. 
\begin{equation*}
(AB)^T=B^TA^T
\end{equation*}
Our equation now becomes 
\begin{equation}
A^T=(P^{-1})^TDP^T
\end{equation}
which equals
\begin{equation}
P'D(P^{-1})'
\end{equation}
This means $A^T$ has n linearly independent eigenvectors if A does as well. 

















\end{document}