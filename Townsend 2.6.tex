\documentclass{article}
\usepackage{amssymb,amsmath,chngcntr,graphicx,float,biblatex,tikz,physics}

\begin{document}
\author{Timothy Smith}
\title{Townsend}
\maketitle

\counterwithin*{equation}{section}
\section*{Problem 2.6}
Evaluate $\hat{R} (\theta \vb{j})\ket{+z}$ where $\hat{R} (\theta \vb{j})= e^{-i\hat{J_z}\theta/\hbar}$ is the operator that rotates kets counterclockwise by the angle $\theta$ about the y axis. Show that $\hat{R} (\frac{\pi}{2} \vb{j})\ket{+z} = \ket{+x}$ Suggestion: Express $\ket{+z} $ as a superposition of the kets $\ket{+y} $ and $\ket{-y}$ and take advantage of the face that $\hat{J_y}\ket{\pm y}=(\pm \frac{\hbar}{2}\ket{\pm y}$; then switch back to the $\ket{+z}-\ket{-z}$ basis. 
\\
\\
Following Townsend's suggestion we need to find  $\ket{+z}$ in the $S_y$ basis. which means finding 
$\braket{+y}{+z}$ and $\braket{-y}{+z}$.
First we need $\bra{+y}$ and $\bra{-y}$ as a superposition of $\bra{\pm z}$
\\
\\
$$ \bra{+y}=\frac{1}{\sqrt{2}}(\bra{+z}-i\bra{-z})$$
$$ \bra{-y}=\frac{1}{\sqrt{2}}(\bra{+z}+i\bra{-z})$$
\\
Solving for $\braket{+y}{+z}$ and $\braket{-y}{+z}$ as follows
\begin{equation}
\braket{+y}{+z}
=
\frac{1}{\sqrt{2}} 
\begin{pmatrix}
1  & -i
\end{pmatrix}
\begin{pmatrix}
1  \\
0
\end{pmatrix}
=\frac{1}{\sqrt{2}}
\end{equation}
\\
and
\\
\begin{equation}
\braket{-y}{+z}
=
\frac{1}{\sqrt{2}} 
\begin{pmatrix}
1  & i
\end{pmatrix}
\begin{pmatrix}
1  \\
0
\end{pmatrix}
=\frac{1}{\sqrt{2}}
\end{equation}
\\
thus $ \ket{+z}$ in an $S_y$ basis is 
\begin{equation}
\ket{+z}=\frac{1}{\sqrt{2}}(\ket{+y}+\ket{-y})
\end{equation}
\\
Taking advantage of the fact that $\hat{J_y}\ket{\pm y}=(\pm \frac{\hbar}{2}\ket{\pm y}$ we can write our rotation operator as 
$$ \hat{R} (\theta \vb{j})=e^{\mp i\frac{\theta}{2}}$$
Applying the rotation operator to $\ket{+z}$ 
\begin{equation}
\hat{R}(\theta\vb{j})\ket{+z}
=
\frac{1}{\sqrt{2}}
\begin{pmatrix}
e^{- i\frac{\theta}{2}} \\
e^{ i\frac{\theta}{2}}
\end{pmatrix}
\end{equation}
Using $\pi/2$ as our $\theta$ our equation is now

\begin{equation}
\hat{R}(\frac{\pi}{2}\vb{j})\ket{+z}
=
\frac{1}{\sqrt{2}}(e^{- i\frac{\pi}{4}}\ket{+y}+e^{ i\frac{\pi}{4}}\ket{-y})
\end{equation}
factoring out a $e^{- i\frac{\pi}{4}}$ our equation becomes
\begin{equation}
\hat{R}(\frac{\pi}{2}\vb{j})\ket{+z}
=
\frac{1}{\sqrt{2}}e^{- i\frac{\pi}{4}}(\ket{+y}+e^{ i\frac{\pi}{2}}\ket{-y})
\end{equation}
which equals 

\begin{align}
=&\frac{1}{\sqrt{2}}(\frac{1}{\sqrt{2}}-\frac{i}{\sqrt{2}})(\ket{+y}+i\ket{-y})
\\
=&(\frac{1}{2}-\frac{i}{2})(\ket{+y}+i\ket{-y})
\\
=&(\frac{1}{2}-\frac{i}{2})(\ket{+y}+i\ket{-y})
\\
=&(\frac{1}{2})(1-i\ket{+y}1+i\ket{-y})
\end{align}
\\
In matrix form our equation in a y basis is 
\begin{equation}
\hat{R}(\frac{\pi}{2}\vb{j})\ket{+z}
=
\frac{1}{2}
\begin{pmatrix}
1-i  \\
1+i 
\end{pmatrix}
\end{equation}
In problem $2.4$ done in class we had previous found 
\begin{equation}
\ket{+x}
\xrightarrow[S_y]{}
\frac{1}{2}
\begin{pmatrix}
1-i  \\
1+i 
\end{pmatrix}
\end{equation}
Thus  $\hat{R}(\frac{\pi}{2}\vb{j})\ket{+z}=\ket{+x}$

















\end{document}