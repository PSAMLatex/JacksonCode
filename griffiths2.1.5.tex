\documentclass{article}
\usepackage{amssymb,amsmath,chngcntr,graphicx,float,biblatex}
\usepackage{calligra}
\DeclareMathAlphabet{\mathcalligra}{T1}{calligra}{m}{n}
\DeclareFontShape{T1}{calligra}{m}{n}{<->s*[2.2]callig15}{}


\begin{document}
\author{Timothy Smith}
\title{E and M}
\maketitle

\newcommand{\scripty}[1]{\ensuremath{\mathcalligra{#1}}}

\counterwithin*{equation}{section}
\section*{Problem 2.1.5 Griffiths }

Find the electric field a distance z above the center of a circular loop of radius r that carries uniform line charge $\lambda$. 
\\
\\
The separation vector is defined as  $\scripty{r}=r-r'$ which in this case equals $z\hat{z}-r\hat{s}$. Thus $\hat{\scripty{r}}$ equals 
$$ \frac{z\hat{z}-r\hat{s}}{\sqrt{z^2+r^2}}$$
and finally dl is
$$rd\phi$$
Thus our electric field equation is 
\begin{equation}
 \vec{E}(\vec{r})
=
\frac{r\lambda}{4\pi\epsilon_0}\int_{0}^{2\pi}\frac{z\hat{z}-r\hat{s}}{(z^2+r^2)^{3/2}}d\phi
\end{equation}
Due to the symmetry of the ring the s portion of our electric field cancels out and thus our equation becomes
\begin{equation}
 \vec{E}(\vec{r})
=
\frac{r\lambda}{4\pi\epsilon_0}\int_{0}^{2\pi}\frac{z\hat{z}}{(z^2+r^2)^{3/2}}d\phi
\end{equation}
Since nothing in our integral depends on $\phi$
Our solution is 
\begin{equation}
 \vec{E}(\vec{r})
=
\frac{r\lambda}{2\epsilon_0}\frac{z\hat{z}}{(z^2+r^2)^{3/2}}
\end{equation}














\end{document}
